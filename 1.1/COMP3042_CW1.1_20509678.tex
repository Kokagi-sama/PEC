\documentclass[a4paper]{article}
\usepackage{geometry}
\geometry{a4paper, left=1in, right=1in, top=1in, bottom=1in}
\usepackage{fancyhdr}
\usepackage{lastpage}
\usepackage{xcolor}
\usepackage[colorlinks=true,
            linkcolor=black,
            urlcolor=blue,
            citecolor=black]{hyperref}
\usepackage{enumitem}
\usepackage[backend=biber]{biblatex} % Use biblatex instead of natbib

\pagestyle{fancy}
\fancyhf{}
\fancyhead[R]{{Coursework 1.1}}
\fancyfoot[L]{{Royceton Yeoh Tze Jian \texttt{|} 20509678}}
\fancyfoot[R]{{Page \thepage\ of \pageref{LastPage}}}

\title{\textbf{Ethical Analysis on Employee Monitoring: The Case of Workday}}
\author{} % Author is left empty as per original document
\date{} % Date is left empty as per original document

\begin{document}

\maketitle

Workday is a cloud-based software platform that helps organisations manage human capital and financial operations. Workday brings together real-time analytics and workforce management in a single platform, allowing employers to track employee behaviours and productivity more closely, dissimilar to older enterprise system such as Oracle. This capability has become particularly appealing in the era of remote work, where monitoring employee engagement and performance is challenging yet essential. Workday’s comprehensive data collection and analytical features make it a relevant subject for \textit{ethical analysis}, particularly regarding \textit{privacy}, \textit{data security}, and the impact of \textit{surveillance} on employee \textit{autonomy} and \textit{agency}.

Workday provides tools for monitoring various aspects of employee performance, from tracking attendance to analysing productivity trends. While these features are intended to help organisations improve efficiency and make better decisions, they also raise concerns about the extent of \textit{surveillance} employees experience. As Workday captures granular details of employee activities, the line between appropriate monitoring and invasive \textit{surveillance} can become blurred.

For instance, constant tracking might lead employees to feel as though they are under a microscope, which can impact their mental well-being and productivity, causing them to prioritise specific metrics over innovation. This is especially relevant in roles that thrive on \textit{innovation} and \textit{collaboration}, where a sense of \textit{autonomy} is crucial for job satisfaction and performance.

One of Workday’s notable capabilities is its use of algorithms to support decision-making processes, such as identifying potential leaders, evaluating performance, and even forecasting turnover. While this can help reduce human biases and improve consistency, it also introduces new \textit{ethical challenges}. Algorithms are only as fair as the data they are trained on, and if past data includes \textit{biases}, the algorithm might reinforce those patterns without any corrective measures. This could impact hiring or promotion decisions, potentially resulting in unintended \textit{discrimination} or \textit{unfair} treatment of employees.

Furthermore, Workday’s data-driven approach can feel opaque to employees, who might not fully understand how decisions about their careers are being made. If employees lack insight into how they are evaluated, they may feel powerless or unfairly judged. Businesses incorporating Workday must consider how to balance the efficiency of \textit{algorithmic decision-making} with the need for \textit{transparency} and \textit{fairness}.

Workday stores large volumes of employees' data and manages it within the organisation. While this can streamline HR processes, it also raises concerns about \textit{data security} and the concentration of control. When sensitive information is stored centrally, it becomes a valuable target for \textit{cyberattacks}, which could expose employee data to misuse. Given the sensitive nature of HR data, any breach could have significant consequences for both the organisation and its employees.

Moreover, this centralisation can also impact the organisation’s culture. Employees might feel uneasy knowing that their personal data is stored and accessible in one place, where decisions about them are made based on extensive monitoring. This raises important questions about data access, security and management, especially in light of privacy regulations such as \textit{GDPR}.

Workday’s capabilities for workforce monitoring and predictive analytics offer organisations considerable advantages in terms of \textit{efficiency} and \textit{insight}, but they also bring ethical challenges to the forefront. The platform’s features make it a timely subject for \textit{ethical scrutiny}, offering a lens into how digital tools can transform workplace dynamics while also posing new risks. 

\thispagestyle{fancy}
\pagestyle{fancy}

\end{document}

