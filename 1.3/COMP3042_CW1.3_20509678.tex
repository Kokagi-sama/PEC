\documentclass[a4paper]{article}
\usepackage{geometry}
\geometry{a4paper, left=1in, right=1in, top=1in, bottom=1in}
\usepackage{fancyhdr}
\usepackage{lastpage}
\usepackage{xcolor}
\usepackage[colorlinks=true,
            linkcolor=black,
            urlcolor=blue,
            citecolor=black]{hyperref}
\usepackage{enumitem}
\usepackage{natbib}

\pagestyle{fancy}
\fancyhf{}
\fancyhead[R]{{Coursework 1.3}}
\fancyfoot[L]{{Royceton Yeoh Tze Jian \texttt{|} 20509678}}
\fancyfoot[R]{{Page \thepage\ of \pageref{LastPage}}}

\title{\textbf{Ethical Framework Proposal on Employee Monitoring: The Case of Workday}}
\author{} % Author is left empty as per original document
\date{} % Date is left empty as per original document

\begin{document}
\maketitle
\thispagestyle{fancy}
\pagestyle{fancy}
\section*{Introduction}
Workday, a work monitoring software, is increasingly used in organisations to track productivity and manage workflows. This proposal applies Kantianism, utilitarianism, and virtue ethics to analyse the ethical implications of its use, focusing on the impact on key stakeholders such as employees, employers, HR departments, and clients. These theories provide a balanced evaluation, addressing both rights and outcomes, as well as the underlying values of the organisation.

\section*{Kantianism}
Kantianism revolves around duties and the inherent rights of individuals, making it essential for evaluating whether Workday respects employee privacy. This theory is crucial for understanding whether monitoring is ethically acceptable as a practice in itself.
My research uncovered a tension between the productivity improvements that Workday offers and the psychological impact on employees, such as feelings of anxiety and pressure due to constant observation. From a utilitarian standpoint, the analysis would assess whether the software’s use leads to a greater overall good. For employers, the data provided by Workday could result in more efficient task allocation and performance management, ultimately leading to improved company performance. However, if the negative consequences for employees such as \textit{stress}, \textit{burnout}, or \textit{attenuated job satisfaction} outweigh the productivity gains, then the use of Workday may not be ethically justified under utilitarian principles. This analysis helps stakeholders like HR determine whether the software's overall impact supports the greater good or causes more harm than benefit, offering a \textit{pragmatic} view of its ethical standing.

\section*{Utilitarianism}
Utilitarianism assesses the consequences of actions, aiming to maximise benefits and minimise harm. It is suitable for evaluating the broader impacts of Workday on stakeholders, including productivity gains for employers and potential stress for employees.
Research findings revealed a tension between increased productivity and negative effects like heightened stress among employees due to constant oversight. Utilitarianism considers whether the positive outcomes, such as efficiency improvements, outweigh the mental and \textit{emotional strain} on employees, \textit{stifling innovation}. For HR and management, this theory helps determine if the overall increase in productivity justifies \textit{potential discomfort}, offering a balanced perspective on whether the software’s implementation leads to a net positive outcome.

\section*{Virtue Ethics}
Virtue ethics examines the character and intentions behind actions, focusing on whether decisions reflect virtues such as trust, fairness, and integrity within the organisation. This theory considers the moral character of the organisation and how the use of Workday aligns with promoting a positive workplace culture. It is relevant for assessing whether monitoring practices foster or undermine a culture of trust between employers and employees.
Insights from my research suggested that the implementation of Workday often reflects an organisation’s values and intentions toward its employees. For instance, if managers use Workday transparently and with the intention to provide constructive feedback or identify areas for employee support, it could align with virtues like fairness and respect. This approach can foster a \textit{sense of mutual trust} and a \textit{collaborative work environment}. However, if Workday is perceived as a means of \textit{exerting control} or \textit{micromanaging} employees, it may create a culture of suspicion and fear, weakening the bond between staff and management. Virtue ethics helps to assess whether the organisation’s use of Workday is driven by a genuine desire to support employee development or merely to enforce \textit{compliance}. This perspective emphasises that the ethicality of using monitoring tools depends on the values that guide their implementation, highlighting the importance of intention and organisational character.

\section*{Conclusion}
Applying Kantianism, utilitarianism, and virtue ethics provides a comprehensive framework for analysing the ethical implications of Workday. Kantianism emphasises employee rights, utilitarianism balances benefits and harms, and virtue ethics focuses on fostering a positive organisational culture. Together, these theories ensure a balanced evaluation of whether the use of Workday is ethically justified, considering both the rights of individuals and the broader impact on stakeholders.


\end{document}