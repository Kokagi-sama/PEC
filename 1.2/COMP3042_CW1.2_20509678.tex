\documentclass[a4paper, 11pt]{article}
\usepackage{geometry}
\geometry{a4paper, left=1in, right=1in, top=1in, bottom=1in}
\usepackage{fancyhdr}
\usepackage{lastpage}
\usepackage{xcolor}
\usepackage[colorlinks=true,
            linkcolor=black,
            urlcolor=blue,
            citecolor=black]{hyperref}
\usepackage{enumitem}
\usepackage{natbib}

\pagestyle{fancy}
\fancyhf{}
\fancyhead[R]{{Coursework 1.2}}
\fancyfoot[L]{{Royceton Yeoh Tze Jian \texttt{|} 20509678}}
\fancyfoot[R]{{Page \thepage\ of \pageref{LastPage}}}

\title{\textbf{Literature Review on Ethical Concerns in Employee Monitoring: The Case of Workday}}
\author{} % Author is left empty as per original document
\date{} % Date is left empty as per original document

\begin{document}
\maketitle
\thispagestyle{fancy}
\pagestyle{fancy}
\section*{Introduction}
Employee monitoring technologies, such as Workday, have transformed workplace management by leveraging AI and analytics. However, these tools raise ethical concerns regarding \textbf{privacy}, \textbf{autonomy}, and \textbf{fairness} in data-driven decision-making. This review analyses literature on these ethical challenges, applying insights from the \textbf{Software Engineering Code of Ethics (SECE)} and ethical theories, including \textbf{Kantianism}, \textbf{Utilitarianism}, and \textbf{Virtue Ethics}.

\section*{Privacy and Consent}
Privacy issues are central to discussions around employee surveillance, as these tools track productivity, location, and biometric data. According to \citet{Moore2018}, such extensive data collection can lead to “datafication” of employees, reducing them to mere metrics on a dashboard. \textbf{SECE Principle 1.05} calls for cooperation in addressing public concerns, highlighting \textbf{transparency about data practices} to uphold trust. \citet{Ajunwa2017} argues that pervasive monitoring often extends into employees' personal lives, creating \textbf{“constant scrutiny”} that intrudes upon private boundaries. \textbf{Kantianism} supports respecting autonomy, highlighting the need for informed consent to protect employees’ rights. \textbf{Virtue Ethics} also aligns with \textbf{SECE Principle 1.06}, which discourages deception, underscoring that fostering a culture of honesty and transparency is essential to maintaining trust between employer and employee.

\section*{Transparency and Accountability}
Transparent monitoring policies are key to fostering trust, yet many companies lack clarity in data collection practices. \textbf{SECE Principle 5.06} stresses the importance of full and accurate information for employees, directly addressing issues raised by \citet{Alge2006} on employees’ lack of knowledge of the extent of monitoring and its impact on trust. \textbf{Virtue Ethics} underscores integrity, affirming that honest disclosure supports a respectful workplace culture. From a \textbf{Utilitarian} standpoint, \textbf{transparency} is beneficial as it positively influences \textbf{morale}, leading to \textbf{increased productivity}, as employees who feel respected and informed are more likely to remain motivated.

\section*{Impact on Employee Autonomy}
Employee surveillance limits autonomy by restricting employees' freedom to self-manage, with \textbf{SECE Principle 1.04} urging employers to disclose such impacts on well-being. Continuous monitoring, as \citet{Sewell2012} notes, can stifle creativity by pressuring adherence to metrics, while \textbf{SECE Principle 6.01} and \textbf{Virtue Ethics} advocate for an ethical environment that supports intrinsic motivation. Platforms like Workday may further drive burnout and disengagement as employees adapt to \textbf{automated metrics} rather than engaging in autonomous, meaningful work as supported by \citep{Gerten2019}.

\section*{Fairness and Bias in Data-Driven Decision-Making}
Bias in surveillance and monitoring systems poses a significant challenge, as data-driven decisions may perpetuate inequalities. \textbf{SECE Principle 3.13} calls for ethical sensitivity, particularly regarding \textbf{bias} in data analysis and decision-making. \citet{Zuboff2019} highlights how AI-based surveillance, often built on historical data, can reinforce existing biases in workplace decisions. \textbf{Kantianism} underscores fairness as a moral duty, advocating unbiased treatment for all employees. \textbf{Virtue Ethics} resonates with\textbf{SECE Principle 7.05}, which promotes fair hearings, arguing that equal and unbiased treatment fosters a supportive and morally sound organisational culture.

\section*{Legal and Ethical Compliance}
Legal compliance is essential in surveillance practices to protect employee rights. \textbf{SECE Principle 1.03} accentuates data safety, aligning with privacy laws such as \textbf{GDPR}, which mandates explicit consent before collecting and using personal data. \citet{Zuboff2019} cautions against bypassing privacy laws, as this can compromise employees’ rights. \textbf{Kantianism} advocates for legal adherence as an extension of respecting autonomy. \textbf{Virtue Ethics}, along with \textbf{SECE Principle 6.06}, reinforces that lawful conduct reflects an organisation’s commitment to ethical principles, which is crucial to building trust between employees and employers.

\section*{Conclusion}
The literature identifies significant ethical challenges in employee surveillance, particularly concerning privacy, autonomy, and fairness. \textbf{Kantianism}, \textbf{Virtue Ethics}, and \textbf{SECE} principles highlight the need for transparency, privacy, and fairness. While \textbf{Utilitarian} emphasise the need for surveillance benefits to outweigh potential harm, Kantianism stresses respecting \textbf{employees' autonomy and rights}. Implementing \textbf{transparent policies} that respect employee rights is essential to maintaining effective and ethical monitoring practices in the modern workplace.

\bibliographystyle{apa}
\bibliography{reference}

\end{document}