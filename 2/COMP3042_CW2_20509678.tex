\documentclass[a4paper]{article}
\usepackage{geometry}
\geometry{a4paper, left=1in, right=1in, top=1in, bottom=1in}
\usepackage{fancyhdr}
\usepackage{lastpage}
\usepackage{xcolor}
\usepackage[colorlinks=true,
            linkcolor=black,
            urlcolor=blue,
            citecolor=black]{hyperref}
\usepackage{enumitem}
\usepackage{natbib}

\pagestyle{fancy}
\fancyhf{}
\fancyhead[R]{{Coursework 2}}
\fancyfoot[L]{{Royceton Yeoh Tze Jian \texttt{|} 20509678}}
\fancyfoot[R]{{Page \thepage\ of \pageref{LastPage}}}

\title{\textbf{Ethical Analysis of Employee Monitoring: The Case of Workday}}
\author{}
\date{}

\begin{document}

\begin{titlepage}
    \centering
    \vspace*{\fill} % Center vertically on the page
    \Huge
    \textbf{Ethical Analysis of Employee Monitoring: The Case of Workday}

    \vspace{1.5cm}
    \LARGE
    Royceton Yeoh Tze Jian \\
    Student ID: 20509678

    \vspace*{\fill} % Center vertically on the page
\end{titlepage}

\thispagestyle{fancy}
\pagestyle{fancy}

\newpage
\tableofcontents
\newpage

\section{Introduction}
Workday is a cloud-based human capital management platform widely used for employee monitoring and decision-making. Designed to improve operational efficiency, Workday provides employers with real-time insights into employee productivity and behaviour. Its popularity, especially with the rise of remote work, highlights its significance in today’s tech landscape. However, Workday’s capabilities raise ethical concerns around privacy, autonomy, and potential biases in algorithmic decision-making. With its extensive use, Workday offers a unique view into the ethical challenges of workplace surveillance \citep{Ajunwa2017, Zuboff2019}.

\section{Involved Stakeholders}
Workday’s employee monitoring capabilities impact multiple stakeholders, each facing ethical challenges concerning privacy, accountability, and trust. These stakeholders include employees, employers, HR departments, clients, and third parties who interact with Workday data. Examining each group reveals the complexities involved in balancing organisational efficiency with ethical considerations.

\subsection{Employees}
Workday’s continuous monitoring significantly impacts employee autonomy and privacy, as it tracks metrics like productivity, time management, and engagement. This constant surveillance can create a stressful work environment, leading to self-censorship and reduced creativity, as employees may feel pressured to conform to monitored behaviours. The opacity in Workday’s data collection and usage practices further intensifies these concerns, as employees often lack clarity on how their data is handled, which can undermine trust and morale. Additionally, using predictive analytics for decisions on promotions or discipline may unintentionally reinforce rigid metrics that don’t accurately capture an individual’s contributions, potentially reducing motivation \citep{Moore2018, Gerten2019}.

\subsection{Employers}
Workday’s monitoring capabilities provide employers with valuable insights into employee performance, helping to optimise productivity and streamline HR processes. These data-driven insights can offer a competitive advantage, enabling informed decisions based on real-time analytics. However, excessive reliance on monitoring raises ethical concerns, as it risks prioritising output over employee well-being, potentially leading to burnout and high turnover. Employers should balance productivity metrics with attention to employee satisfaction and mental health, ensuring that monitoring practices support both organisational goals and a positive workplace culture \citep{Sewell2012}.

\subsection{HR Departments}
HR departments play a key role in the ethical deployment of Workday by managing data accuracy, addressing potential biases in decision-making, and ensuring compliance with regulations like GDPR. As primary interpreters of employee data, HR must guard against unfair patterns, particularly if predictive algorithms rely on biased historical data. Additionally, HR should ensure that data is stored securely, accessed only for legitimate purposes, and used in ways that respect privacy rights. Transparent data practices foster trust and help align Workday’s monitoring capabilities with ethical policies that support employee welfare and organisational integrity \citep{Alge2006}.

\subsection{Clients and Third Parties}
Clients and third-party vendors accessing Workday’s data bear ethical responsibilities, especially around data security and regulatory compliance. Clients must ensure that monitoring practices align with ethical standards and legal regulations like GDPR, considering impacts on employee morale and privacy. Third-party vendors, such as IT providers, must uphold strict data protection standards to prevent misuse. These stakeholders share the responsibility of securing sensitive employee information, fostering a workplace that values privacy, accountability, and respect for personal data \citep{Ajunwa2017}.

\section{Analysis of Ethical Issues Using SE Code of Ethics}
The ethical concerns in Workday’s monitoring practices can be examined through the Software Engineering Code of Ethics (SECE). This analysis uses relevant SECE themes—such as Public Interest, Judgment, and Management—to understand how Workday’s capabilities align with or diverge from accepted ethical standards in software engineering.

\subsection{Public Interest (SECE Principles 1.02, 1.04, 1.06)}
A central ethical issue in Workday’s monitoring system is the potential infringement on employee privacy, which is integral to public interest. SECE Principle 1.02, which encourages professionals to balance competing interests, supports the need for a nuanced approach to employee monitoring, where productivity benefits do not overshadow the fundamental right to privacy. By weighing these interests, Workday can achieve organisational efficiency while respecting employee autonomy. SECE Principle 1.04 reinforces the need for transparency by advising professionals to disclose any actual or potential dangers to the public, including privacy risks. In the context of Workday, disclosing data usage policies and any potential implications for privacy helps employees make informed decisions and fosters trust. Principle 1.06 advocates for honesty and fairness in all professional statements, which implies that employees should receive clear, accurate information about how monitoring data is used, shared, and retained. By adhering to these principles, Workday can promote a culture of openness and accountability, addressing public interest concerns and mitigating potential privacy risks \citep{Zuboff2019}.

\subsection{Judgment (SECE Principles 4.01, 4.03)}
The SECE principles emphasise the importance of sound professional judgment, particularly when balancing technical functionality with human values. SECE Principle 4.01 advises that all technical judgments should be tempered by the need to maintain and support human values, such as respect for privacy and well-being. For employers using Workday, this principle calls for careful consideration of how continuous monitoring might impact employees psychologically, avoiding approaches that may inadvertently create stress or anxiety. SECE Principle 4.03 highlights the importance of objectivity, urging transparency in decisions that affect workplace culture. In practice, this means that organisations should provide clear rationales for the use of monitoring tools, ensuring employees understand why and how monitoring data impacts their evaluations or work environment. This objective and balanced approach helps preserve employee morale and supports a respectful, positive workplace atmosphere \citep{Ajunwa2017, Moore2018}.

\subsection{Product (SECE Principles 3.12, 3.13)}
As a software product, Workday should adhere to SECE’s product-focused principles, which stress privacy, fairness, and the responsible use of technology. SECE Principle 3.12 promotes the development of software that respects the privacy of those affected by it, an essential consideration in managing sensitive employee data within Workday’s platform. This principle implies that Workday should incorporate privacy-by-design approaches, ensuring that data collection and usage are transparent and limited to what is necessary. SECE Principle 3.13 stresses the importance of using only accurate, appropriately acquired data, which reduces the likelihood of bias in performance assessments. This principle advocates for rigorous data validation and fair practices in algorithmic decision-making, ensuring that employees are not unfairly evaluated due to inaccuracies or poorly acquired data. Adherence to these product standards not only supports a fair and transparent workplace environment but also strengthens the integrity and reliability of the monitoring system itself \citep{Sewell2012}.

\subsection{Colleagues (SECE Principles 7.05, 7.04)}
The SECE principles concerning colleagues underscore the importance of fair treatment, mutual respect, and support within the workplace. SECE Principle 7.05 encourages professionals to give colleagues a fair hearing, emphasising the need for impartiality in monitoring practices. In Workday’s context, this principle supports the idea that employee data should not be used to unfairly favour or disadvantage certain individuals. Rather, monitoring data should be applied in an equitable manner that respects each employee’s unique contributions. SECE Principle 7.04 advises that evaluations of colleagues’ work be objective, a crucial consideration when interpreting data derived from monitoring systems. By ensuring that performance reviews and evaluations based on Workday’s data remain unbiased, organisations can foster an environment of trust and respect among employees, supporting a collaborative and positive workplace culture \citep{Alge2006}.


\section{Application of Selected Ethical Theories}
Ethical theories offer additional perspectives on Workday’s monitoring, particularly in balancing productivity with employee rights, while counteracting theories provide a critical view of potential ethical shortcomings.

\subsection{Kantianism}
Kantian ethics, which values respect for individual autonomy and dignity, would critique Workday’s monitoring if it compromises employee rights to privacy. Kantian principles call for transparent data handling and giving employees control over their data to uphold moral rights. However, Natural Law Theory offers a counterpoint, suggesting that privacy is an inherent right that should not be infringed upon, regardless of organisational benefits. From this perspective, monitoring that prioritises productivity at the expense of privacy may contravene the duty to respect fundamental human dignity, requiring that Workday’s practices align more closely with respecting inherent moral values \citep{Alge2006}.

\subsection{Utilitarianism}
Utilitarianism justifies monitoring if it maximises overall benefit without disproportionate harm to employees, supporting productivity gains so long as they are balanced with employee well-being. For Workday, this approach means enhancing productivity while safeguarding employee welfare through transparent policies and data protections. Yet, the Ethics of Care offers a contrasting view by prioritising empathy and nurturing relationships, suggesting that constant surveillance can strain trust and induce stress. This theory argues that the well-being of employees should not be sacrificed for organisational gain, emphasising a compassionate approach to monitoring that respects employee morale and mental health \citep{Gerten2019}.

\subsection{Virtue Ethics}
Virtue Ethics, focusing on organisational character and virtues like honesty, trust, and respect, promotes transparency and fair data practices in Workday’s monitoring approach. By aligning with virtues such as integrity, Workday can foster a respectful workplace culture. However, the Ethic of Justice challenges whether monitoring practices equally protect all employees, as bias in algorithmic data processing can unfairly impact certain individuals. This theory advocates for equity and fairness, urging that Workday’s data practices should be free from biases to avoid compromising organisational integrity and the fair treatment of all employees \citep{Moore2018, Sewell2012}.

\section{Critical Reflection}
Reflecting on Workday’s practices through SECE and ethical theories reveals the nuanced ethics of employee monitoring. SECE provides structured guidelines for privacy, fairness, and accountability, while Kantianism and Virtue Ethics focus on respecting individual dignity. Utilitarianism offers a practical view, balancing productivity with well-being, suggesting that no single framework fully addresses Workday’s ethical challenges.

\subsection{SECE: Structured Guidance}
SECE offers structured guidance on privacy, fairness, and accountability, underscoring the need for transparent data policies and unbiased treatment. Workday’s compliance with SECE standards ensures that productivity goals are balanced with employee rights, providing actionable guidance for organisations \citep{Zuboff2019}.

\subsection{Kantianism and Virtue Ethics: Protecting Dignity}
Kantianism and Virtue Ethics emphasise respect for individual rights, advocating transparency in data practices to uphold dignity and integrity. Workday’s transparency reports and privacy policies are not mere procedures but are essential to maintaining a respectful workplace culture, as these theories see ethical monitoring as a reflection of core organisational values.

\subsection{Utilitarianism: Balancing Productivity and Well-being}
Utilitarianism supports monitoring if it maximises collective benefits, advocating transparency and minimising intrusive surveillance. Transparent data handling and strong privacy policies help balance organisational success with employee well-being, aligning monitoring practices with both productivity and mental health.

\subsection{A Multi-layered Approach}
No single framework can fully address Workday’s ethical complexities. SECE offers a foundation, while Kantianism and Virtue Ethics prioritise respect, and Utilitarianism balances goals with individual welfare. This multi-layered approach reveals the importance of adaptable ethics that respect individual rights, foster trust, and align productivity with well-being.

\section{Recommendations for Improving Ethical Standing}
Based on the analysis, the following recommendations are proposed:
\begin{itemize}
    \item \textbf{Increase Transparency:} Workday should enhance data transparency by clarifying data collection, usage, and retention, aligning with SECE and Virtue Ethics.
    \item \textbf{Consent and Autonomy:} Workday should enable informed employee choice regarding data sharing, aligning with Kantian respect for autonomy.
    \item \textbf{Mitigate Algorithmic Bias:} Regular audits should address biases, aligning with SECE Principle 3.13 and ensuring fair treatment.
    \item \textbf{Focus on Ethical Data Usage:} Data practices should prioritise employee well-being, fostering a supportive environment aligned with the Ethics of Caring.
\end{itemize}
These recommendations align Workday’s practices with SECE and ethical theories, promoting respectful and balanced monitoring practices.

\section{Conclusion}
Workday’s monitoring capabilities present benefits for organisational efficiency but raise ethical concerns around privacy, autonomy, and fairness. SECE and ethical theories such as Kantianism, Utilitarianism, and Virtue Ethics offer diverse approaches to these challenges. Implementing the proposed recommendations can help Workday balance productivity with ethical integrity, fostering a transparent, respectful workplace that upholds employee dignity and data security.

\bibliographystyle{apa}
\bibliography{references}

\end{document}
